\chapter*{Conclusions and future developments}
\noindent This thesis work was performed within MoVeIT, a research
project by the Medical Physics group at University of Turin
and INFN, which aims to develop new and
innovative models for biologically optimized Treatment
Planning Systems (TPS) using ion beams in hadron therapy.
The particles are detected using a Low-Gain Avalanche Diode
(LGAD) thin silicon sensor segmented in strips. The signal
is then read out by a full-custom ASIC designed by Turin
INFN, named ABACUS2.\\
\noindent This thesis reported my personal contributions on the
implementation and validation of new additions to the
FPGA firmware used to characterize and test the ABACUS2
chip.\\
\noindent The work was carried out using VHDL language and
Xilinx Vivado software on Xilinx Kintex 7 KC705 FPGA boards. 
I personally implemented:
i) the logic for the configuration of the ABACUS2 trimming
DACs, that now use an I2C based address system;
ii) the creation of a latch module that saves into BRAM the
value of every FPGA counter at the same time;
iii) the implementation of a timestamp generator in order
to obtain a more accurate measurement of the particle rate.
Moreover, I developed a debug tool for the verification
of the FPGA channels working conditions.\\
\noindent The results of this thesis are:
i) the ABACUS2 I2C controller and the FPGA configuration
logic for the trimming DACs are working as intended; 
ii) from the performed threshold scans emerged that the ABACUS2
internal DACs are changing the threshold voltage and thus
shifting the S-curves as expected;
iii) the latch system works properly and freezes the state
of each counter into a register;
iv) the new timestamp generator improves the rate
calculation accuracy, lowering the standard deviation by a factor of 90; 
v) the new rate measurement system allows lower integration times
and at the same time keeps the uncertainty under 1\%;
vi) the FMC debug tool can correctly be used for the identification
of malfunctioning channels on the FPGAs board.\\
\noindent Possible future developments of the MoVeIT project
include:
i) the creation of a software configurable mask that allows
the hardware calculation of the sum of the counts of only
the selected channels;
ii) the implementation of the particle rate calculation directly
on board;
iii) the coding of a fully capable data acquisition panel with
LabVIEW able of using at his full extend the new features of the
FPGA firmware.



\chapter*{Acknowledgements}
\vspace*{3cm}
\noindent I would like to dedicate this thesis to my parents, \textit{Cristina Staviri} and \textit{Nicolae Zugravel} who have always supported me during my university career.
They kept me in Turin for five years and never put pressure on me.
A thought of mine goes to my partner \textit{Giorgia Rista} on whom I have always been able to count.
I would like to sincerely thank my supervisor \textit{Luca Pacher} who has always helped me and inspired to do better.
My thanks also go to my co-supervisor Professor \textit{Vincenzo Monaco} who has always been available for all my questions and to researcher \textit{Simona Giordanengo} who helped me a lot during the validation process.
Finally, my thoughts go to all my friends, both those I met at the university and those I met many years before.
Working for this project has been a pleasure and an experience that has taught me a lot.
I can't wait to continue my journey and improve myself more and more.
To conclude, I would like to thank one last time whoever has read this thesis work, I hope you have enjoyed reading it and that you have found it interesting as much as I did.\\
\noindent Sincerely.
\vspace*{1cm}
\begin{flushright}
	\textit{Stefan Cristi Zugravel}
\end{flushright}