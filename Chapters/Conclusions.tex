\chapter*{Conclusions and future developments}
\pagestyle{plain}
\noindent This thesis work was carried out inside the MoVe\_IT project. Along side my work on the FPGA firmware other students were involved on the chip characterisation, test/data-analysis of the ESA-ABACUS board and time of flight measurements.
I personally designed, implemented and tested with success three additions to the FPGA firmware that will be used for hadron therapy tests.
In addition, I developed a debug tool for the verification of the working conditions of the FPGA channels.
My job started with an already working project created by \textit{Richard Wheadon} and updated by \textit{Alessio Limardi}.
For the lab tests I was helped by INFN researcher \textit{Simona Giordanengo} and thesis student \textit{Emanuele Data}.




\section*{Greetings}
I would like to dedicate this thesis to my parents, \textit{Cristina Staviri} and \textit{Nicolae Zugravel} who have always supported me during my university career.
They kept me in Turin for five years and never put pressure on me.
A thought of mine goes to my partner \textit{Giorgia Rista} on whom I have always been able to count.
I would like to sincerely thank my supervisor \textit{Luca Pacher} who has always helped me and inspired to do better.
My thanks also go to my co-supervisor Professor \textit{Vincenzo Monaco} who has always been available for all my questions and to researcher \textit{Simona Giordanengo} who helped me for the validation process.
Finally, my thoughts go to all my friends, both those I met at the university and those I met many years before.