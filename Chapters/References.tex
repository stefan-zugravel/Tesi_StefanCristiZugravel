\begin{thebibliography}{00}	
	\bibitem[1]{tumor}www.medicalnewstoday.com/articles/249141
	\bibitem[2]{radiationtherapy}enlight.web.cern.ch/what-is-hadron-therapy
	\bibitem[3]{landau}L. Landau, On the Energy Loss of Fast Particles by Ionization, Journal of
	Physics 8, (1944), 201 -205
	\bibitem[4]{cells}H. Tsujii et al., Carbon-Ion Radiotherapy – Principles, Practices and Treatment
	Planning, Springer, (2012)
	\bibitem[5]{let}D. Schardt, T. Elsasser, Heavy-ion tumor therapy: Physical and radiobiological
	benefits, Reviews of Modern Physics, 82, (2010)
	\bibitem[6]{cnao}Giordanengo et al., Design and characterization of the beam monitor detectors
	of the Italian National Center of Oncological Hadron-therapy (CNAO).
	\bibitem[7]{pencil}S. Giordanengo et al , The CNAO Dose Delivery System for ion pencil beam
	scanning radiotherapy, Medical Physics (2015)
	\bibitem[8]{detector}Hartmann, F., Silicon tracking detectors in high energy physics, Nucl.Instr.Meth.A
	666 (2012) 25-46.
	%----------------------------------------------------------------------
	\bibitem[9]{moveit}MoVe\_IT, Modeling and Verification for Ion beam Treatment planning, INFN CSN5 “Call 2017” Project proposal
	\bibitem[10]{lgad}G. Pellegrini et al, Technology developments and first measurements of Low Gain Avalanche Detectors (LGAD) for high energy physics applications, Nucl.Instrum.Meth. A765 (2014) 12-16.
	\bibitem[11]{abacus}A single ion discriminator ASIC prototype for particle therapy applications
	\bibitem[12]{hammad}CHARACTERIZATION AND TEST OF LGAD STRIP SILICON DETECTORS TO COUNT THE NUMBER OF PROTONS OF THERAPEUTIC BEAMS, Omar Hammad Ali
	%----------------------------------------------------------------------
	\bibitem[13]{fpga1}Introduction to FPGA design, J. Serrano, CERN-Geneva-Switzerland
	\bibitem[14]{fpga2}Three Ages of FPGAs: A Retrospective on the First Thirty Years of FPGA Technology, By Stephen M. (Steve) Trimberger, Fellow IEEE
	\bibitem[15]{fpga3}Introduction to Field Programmable Gate Arrays Hannes Sakulin CERN / EP-CMD
	\bibitem[16]{fpga4}Field Programmable Gate Arrays and Applications, Version 2 EE IIT, Kharagpur
	\bibitem[17]{hdl}en.wikipedia.org/wiki/Hardware\_description\_language
	\bibitem[18]{vhdl}Free Range VHDL, Bryan Mealy, Fabrizio Tappero
	\bibitem[19]{constraints1}Vivado Design Suite User Guide
	\thispagestyle{plain}
	\bibitem[20]{constraints2}Constraints Guide ISE 6.2.03i
	\bibitem[21]{vivado}en.wikipedia.org/wiki/Xilinx\_Vivado
	\bibitem[22]{labview}www.ni.com/pdf/manuals/320999e.pdf
	\bibitem[23]{kintex7}KC705 Evaluation Board for the Kintex-7 FPGA	
	\bibitem[24]{tcl}www.tcl.tk
	\bibitem[25]{lvds}en.wikipedia.org/wiki/Low-voltage\_differential\_signaling
	%----------------------------------------------------------------------
	\bibitem[26]{gbphy}personalpages.to.infn.it/~wheadon/fpga/phy100/default.html
	\bibitem[27]{dac}MoveIt v2 design document, Giovanni Mazza, January 15 2021
	\bibitem[28]{timing}www.eejournal.com/article/20130528-timing/
	\bibitem[29]{limardi}Sviluppo su FPGA di tecniche di correzione di effetti d'inefficienza nel conteggio di singoli protoni in fasci terapeutici, Alessio Limardi, 2019-2020.
	%----------------------------------------------------------------------
	\bibitem[30]{LTC2604}www.analog.com/media/en/technical-documentation/data-sheets/2604fd.pdf
	\bibitem[31]{data}Emanuele Data's masters degree thesis
	\bibitem[32]{rivetti} Angelo Rivetti,
	
	
\end{thebibliography}
