\begin{thebibliography}{00}	
	\bibitem[1]{tumor}www.medicalnewstoday.com/articles/249141
	\bibitem[2]{radiationtherapy}enlight.web.cern.ch/what-is-hadron-therapy
	\bibitem[3]{landau}L. Landau, On the Energy Loss of Fast Particles by Ionization, Journal of
	Physics 8, (1944), 201 -205
	\bibitem[4]{cells}H. Tsujii et al., Carbon-Ion Radiotherapy – Principles, Practices and Treatment
	Planning, Springer, (2012)
	\bibitem[5]{let}D. Schardt, T. Elsasser, Heavy-ion tumor therapy: Physical and radiobiological
	benefits, Reviews of Modern Physics, 82, (2010)
	\bibitem[6]{cnao}Giordanengo et al., Design and characterization of the beam monitor detectors
	of the Italian National Center of Oncological Hadron-therapy (CNAO).
	\bibitem[7]{pencil}S. Giordanengo et al , The CNAO Dose Delivery System for ion pencil beam
	scanning radiotherapy, Medical Physics (2015)
	\bibitem[8]{detector}Hartmann, F., Silicon tracking detectors in high energy physics, Nucl.Instr.Meth.A
	666 (2012) 25-46.
	
	----------------------------------------------------------------------
	
	----------------------------------------------------------------------
	\bibitem[9]{fpga1}Introduction to FPGA design, J. Serrano, CERN-Geneva-Switzerland
	\bibitem[10]{fpga2}Three Ages of FPGAs: A Retrospective on the First Thirty Years of FPGA Technology, By Stephen M. (Steve) Trimberger, Fellow IEEE
	\bibitem[11]{fpga3}Introduction to Field Programmable Gate Arrays Hannes Sakulin CERN / EP-CMD
	\bibitem[12]{fpga4}Field Programmable Gate Arrays and Applications, Version 2 EE IIT, Kharagpur
	\bibitem[13]{hdl}en.wikipedia.org/wiki/Hardware\_description\_language
	\bibitem[14]{vhdl}Free Range VHDL, Bryan Mealy, Fabrizio Tappero
	\bibitem[15]{constraints1}Vivado Design Suite User Guide
	\bibitem[16]{constraints2}Constraints Guide ISE 6.2.03i
	\bibitem[17]{vivado}en.wikipedia.org/wiki/Xilinx\_Vivado
	\bibitem[18]{labview}www.ni.com/pdf/manuals/320999e.pdf
	\bibitem[19]{kintex7}KC705 Evaluation Board for the Kintex-7 FPGA	
	\bibitem[20]{tcl}www.tcl.tk
	
	
	----------------------------------------------------------------------
	\bibitem[21]{gbphy}personalpages.to.infn.it/~wheadon/fpga/phy100/default.html
	\bibitem[22]{dac}MoveIt v2 design document, Giovanni Mazza, January 15 2021
	\bibitem[23]{timing}www.eejournal.com/article/20130528-timing/
	
	----------------------------------------------------------------------
	
	\bibitem[99]{limardi}Sviluppo su FPGA di tecniche di correzione di effetti d'inefficienza nel conteggio di singoli protoni in fasci terapeutici, Alessio Limardi, 2019-2020.
	
\end{thebibliography}
