\chapter{Hadron Therapy}

\section{Introduction}
The National Cancer Institute define a tumor\cite{tumor} as “an abnormal mass of tissue that results when cells divide more than they should or do not die when they should.”
In a healthy body, cells grow, divide, and replace each other in the body. As new cells form, the old ones die. When a person has cancer, new cells form when the body does not need them. If there are too many new cells, a group of cells, or tumor, can develop.
A tumor develops when cells reproduce too quickly. Tumors can vary in size from a tiny nodule to a large mass, depending on the type, and they can appear almost anywhere on the body.
There are three main types of tumor:
\begin{itemize}
\item \textbf{Benign}: These are not cancerous. They either cannot spread or grow, or they do so very slowly. If a doctor removes them, they do not generally return.
\item \textbf{Premalignant}: In these tumors, the cells are not yet cancerous, but they have the potential to become malignant.
\item \textbf{Malignant}: Malignant tumors are cancerous. The cells can grow and spread to other parts of the body.
\end{itemize}
Radiation therapy is the medical use of ionizing radiation to treat cancer. In conventional radiation therapy, beams of X rays (high energy photons) are produced by accelerated electrons and then delivered to the patient to destroy tumour cells. Using crossing beams from many angles, radiation oncologists irradiate the tumour target while trying to spare the surrounding normal tissues. Inevitably some radiation dose is always deposited in the healthy tissues.
When the irradiating beams are made of charged particles (protons and other ions, such as carbon), radiation therapy is called hadrontherapy\cite{radiationtherapy}. The strength of hadrontherapy lies in the unique physical and radiobiological properties of these particles; they can penetrate the tissues with little diffusion and deposit the maximum energy just before stopping. This allows a precise definition of the specific region to be irradiated. The peaked shape of the hadron energy deposition is called Bragg peak and has become the symbol of hadrontherapy. With the use of hadrons the tumour can be irradiated while the damage to healthy tissues is less than with X-rays.

\section{Interaction between matter and charged particles}


\begin{equation}\label{eq:bethe}
	-\dfrac{\mathrm dE}{\mathrm dx} = 2 \pi N_{a} r_{e}^{2} m_{e} c^{2} \rho \dfrac{Z}{A}  \dfrac{z^{2}}{\beta^{2}}\left[\ln\left(\dfrac{2m_{e} \gamma ^{2} v^{2} W_{max}}{I^{2}}\right) - 2\beta^{2} - \delta 2\frac{C}{Z}\right]
\end{equation}


\section{Effects of radiations on biological systems}

\section{Dose distribution systems in hadron therapy}

\subsection{Passive dose distribution systems}

\subsection{Active dose distribution systems}

\subsection{Treatment Planning System}

\section{Beam monitoring}

\subsection{Silicon detectors}